\documentclass[a4paper,12pt]{article}
\usepackage[utf8]{inputenc}
\usepackage[english]{babel}
\usepackage[T1]{fontenc}
\usepackage{graphicx}

\title{Snake-AI\\Project Documentation}
\author{Ibrahim Enes Hayber, Rana MD Jewel, Maximilian Lüttich\\
		Frankfurt University of Applied Sciences\\
		Faculty 2, Computer Science (B. Sc.)\\
		Object-oriented Programming in Java by Prof. Dr. Doina Logofatu
}

\date{\today}

\begin{document}

\maketitle
\begin{center}
\includegraphics[scale=0.8]{fra-uas-logo}
\end{center}
\newpage
\tableofcontents

\newpage
\section{How we started}
This chapter describes how we started with the project.\\
\\As we all know the hardest part of a journey is the start. But once started it is a easy on going.
It is like the impact on the first domino stone, which brings the whole project in rolling.
\\As usual in any task of life we need to understand what the problem is and which requirements are necessary to solve the problem. To do so, we read the description from the organizers carefully. In addition to that we watched a few videos about our topic to get a better understanding.\\
After gaining knowledge about the problem we felt ready to start working with the Snake-AI framework, provided by the organizers on a GitHub repository. The framework was written in the Java programming language. To get familiar we looked into the UML class
diagram. Besides that we followed a YouTube Tutorial which was also provided by the organizers, where we implemented our first working version. A detailed introduction and our results will now be presented.
\newpage

\section{Introduction}
Many people know the popular game "Snake" which appeared on a Nokia device back in 1997.
This project focuses on reestablishing Snake but in a different way. 
The most special is that instead moving one single snake manually, we will have two intelligent Snake bots competing against each other.
The aim of a bot is to be the last standing snake or have the highest score by eating apples in 3 minutes. 
\textit{Survival of the fittest} is a great saying.\\
This chapter focuses on the general overview of this project and comparison to similar problems in practice.
\subsection{General Topic}

\subsection{Similar problems in practice (References every time, look for actual ones)}
to-do

\section{Team Work}
to-do
\subsection{Work Participation}
to-do

\subsection{work structure (communication, decisions, bug tracking, repository, engineering, ...)}
to-do

\subsection{ideas (brainstorming)}
to-do

\section{Problem Description}
to-do
\subsection{Formal description: definitions, examples, ...}
to-do

\section{Related Work}
to-do
\subsection{Related Algorithms, Applications}
to-do

!for example: Fractals, Data Generation, Games, Evolutionary Algorithms (Genetic Algortihms; Collective Intelligence like Particle Swarm Optimization - PSO, Ant Colony Optimizaton - ACO; Evolutionary Multiobjective Optimization, ...)

\section{Proposed Approaches}
to-do
\subsection{Input/Output Format, Benchmarks (Generation, Examples)}
to-do
\subsection{Algorithms in Pseudocode}
to-do

\section{Implementation Details}
to-do
\subsection{Application Structure}
to-do
\subsection{GUI Details}
to-do
\subsection{UML Diagram}
to-do
\subsection{Used Libraries}
to-do
\subsection{Code Snippets}
to-do
\section{Evolution of Bots}
to-do
\subsection{Bot1}
\subsection{Bot2}
\subsection{Bot3}

\section{Experimental Results and Statistical Tests}
to-do
\subsection{Simulations (Play with the Parameters!)}
to-do
\subsection{Use Benchmarks}
to-do
\subsection{Tables: input data details, results of different algorithms}
to-do
\subsection{Charts}
to-do
\subsection{Evaluations}
to-do

\section{Conclusions and Future Work}
to-do
\subsection{How the team work went}
to-do
\subsection{What we have learned}
to-do
\subsection{Ideas for the future development of your application, new algorithms}
to-do


\end{document}