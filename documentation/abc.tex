\documentclass[a4paper,12pt]{article}
\usepackage[utf8]{inputenc}
\usepackage[english]{babel}
%\documentclass{IEEEtran}
\usepackage[T1]{fontenc}
%\usepackage{apacite}

\title{Snake-AI\\Project Documentation}
\author{Ibrahim Enes Hayber, Rana MD Jewel, Maximilian Lüttich\\
		Frankfurt University of Applied Sciences\\
		Faculty 2, Computer Science (B. Sc.)\\
		Object-oriented Programming in Java by Prof. Dr. Doina Logofatu
}

\date{\today}

\begin{document}

\maketitle
\newpage
\tableofcontents
\newpage

\section{How we started}
This chapter describes how we started with the project.\\
\\As we all know the hardest part of a journey is the start. But once started it is a easy on going.
It is like the starting impact on the first domino stone, which brings the whole project in rolling.
\\As usual in any task of life we need to understand what the problem is and which requirements are necessary to solve the problem. To do so, we read the description from the organizers carefully. In addition to that we watched a few videos about our topic to get a better understanding.\\
After gaining knowledge about the problem we felt ready to start working with the Snake-AI project provided by the organizers on a GitHub repository. The Project was written in the Java programming language. To get familiar with the project we looked into the UML class
diagram. Besides that we followed a YouTube Tutorial which was also provided by the organizers, where we implemented our first working version. A detailed introduction and our results will now be presented.
\newpage

\section{Introduction}
to-do
\subsection{General Topic}
to-do
\subsection{Similar problems in practice (References every time, look for actual ones)}
to-do

\section{Team Work}
to-do
\subsection{Work Participation}
to-do

\subsection{work structure (communication, decisions, bug tracking, repository, engineering, ...)}
to-do

\subsection{ideas (brainstorming)}
to-do

\section{Problem Description}
to-do
\subsection{Formal description: definitions, examples, ...}
to-do
jflafjlajflajfla
\section{Related Work}
to-do
\subsection{Related Algorithms, Applications}
to-do

!for example: Fractals, Data Generation, Games, Evolutionary Algorithms (Genetic Algortihms; Collective Intelligence like Particle Swarm Optimization - PSO, Ant Colony Optimizaton - ACO; Evolutionary Multiobjective Optimization, ...)

\section{Proposed Approaches}
to-do
\subsection{Input/Output Format, Benchmarks (Generation, Examples)}
to-do
\subsection{Algorithms in Pseudocode}
to-do

\section{Implementation Details}
to-do
\subsection{Application Structure}
The Snake AI project consists of one Interface, one Enum and eight classes. They are following, 

\begin{itemize}
\item Bot(Interface)
\item Direction(Enum)
\item BotLoader
\item Coordinate
\item Snake
\item SnakeGame
\item SnakeCanvas
\item SnakeRunner
\item SnakeUIMain
\item SnakesWindow
\end{itemize}

Names of packages that you want to document, separated by spaces, for example java.lang java.lang.reflect java.awt. If you want to also document the subpackages.

\subsection{GUI Details}
to-do
\subsection{UML Diagram}
to-do
\subsection{Used Libraries}
to-do
\subsection{Code Snippets}
to-do

\section{Experimental Results and Statistical Tests}
to-do
\subsection{Simulations (Play with the Parameters!)}
to-do
\subsection{Use Benchmarks}
to-do
\subsection{Tables: input data details, results of different algorithms}
to-do
\subsection{Charts}
to-do
\subsection{Evaluations}
to-do

\section{Conclusions and Future Work}
to-do
\subsection{How the team work went}
to-do
\subsection{What we have learned}
to-do
\subsection{Ideas for the future development of your application, new algorithms}
to-do

\end{document}